\chapter{Conclusion}\label{ch:conclusion}
Carbon markets on the blockchain are an evolving area of research
in a market requiring a high level of `trust' amongst participants.
Current research has primarily focused on the creation of
a carbon market using manual processes, such as permit approvals.
Complicating matters a significant amount of the early research
has focused on modifying existing cyptocurrencies to accomodate
emissions trading on the blockchain. The goal of carbon trading
is to discourage enterprises from emitting carbon,
therefore a permissioned blockchain
with identifiable participants is required. In this thesis, I propose
the use of certificate-based smart contracts as a mechanism to drive
carbon markets. I present the use of the high throughput blockchain
Hyperledger Fabric as a framework for creating an emissions trading
scheme on-chain. A Fabric blockchain allowing hydrogen producers
to validate hydrogen certificates, purchase tokens and trade carbon
permits has been proposed. The primary outcome of the thesis
is the presentation of a technology to deliver transparency and
automation into the market for carbon.