\section{Further Work}
\subsection{Architecture Gaps}
\begin{frame}{Architecture Gaps}
    \begin{itemize}
        \item The focus of the thesis was to create a carbon market
              motivated by automatic ESG certification.
        \item The architecture for an `ESG Channel' researched by Liu et al
              in 2021 needs to be further explored and implemented in detail.
        \item Although payment settlment happens on-chain, further work is
              required to explore how a producer could pay for production in
              a currency outisde of Australian dollars.
              \begin{itemize}
                  \item Trust could be further generated by allowing payment
                        in cryptocurrencies such as \textit{Ether} or \textit{Bitcoin}.
              \end{itemize}
    \end{itemize}
\end{frame}
\subsection{Performance}
\begin{frame}{Transactions Per Second}
    \begin{itemize}
        \item The decision to move the order book on-chain comes at the
              price of a significant performance loss.
        \item Generally, on-chain order books suffer from low TPS
              due to \textit{phantom read conflicts} - a blockchain
              phenomenon where assets are read in the same block of
              transactions where they are written to.
        \item To scale the carbon market for a large number of producers
              a sensible architecture would be having an off-chain order
              book with sale finalisation happening on-chain.
        \item Alternatively, a specialised blockchain and protocol
              such as \textit{Solana} and \textit{Serum} could be used
              at the price of using a more unstable technology compared to
              \textit{Hyperledger}.
        \item The issue of where to put the order book is contentious,
              and depends on the scale the carbon market is expected to reach.
    \end{itemize}
\end{frame}
\begin{frame}{Thank You}
    \begin{itemize}
        \item Questions?
    \end{itemize}
\end{frame}