\chapter{Project Dependent Preparations}\label{ch:prep}
An observation of Table~\ref{tab:technology} shows the
libraries and frameworks I propose to use for creating the
carbon market. My competency with the frameworks outside of
Hyperledger Fabric are high.

\begin{table}[ht]
    \centering
    \begin{tabular}{|l|l|}
        \hline
        \multicolumn{2}{|l|}{\textbf{User Interface}}                   \\ \hline
        Library            & Use Case                                   \\ \hline
        React.js           & \begin{tabular}[c]{@{}l@{}}Component and state drive JavaScript library for \\ making user interfaces.\end{tabular}                  \\ \hline
        Material UI        & Component library.                         \\ \hline
        React Router       & Navigational components.                   \\ \hline
        Styled Components  & CSS in JavaScript.                         \\ \hline
        Cypress            & Integration testing for user interfaces.   \\ \hline
        \multicolumn{2}{|l|}{\textbf{API Layer}}                        \\ \hline
        Library            & Use Case                                   \\ \hline
        DynamoDB           & NoSQL database for storing off-chain data. \\ \hline
        Express            & Back end framework for Node.               \\ \hline
        \multicolumn{2}{|l|}{\textbf{Smart Contract Layer}}             \\ \hline
        Library            & Use Case                                   \\ \hline
        Hyperledger Fabric & \begin{tabular}[c]{@{}l@{}}Modular and interoperable framework for \\ making permissioned blockchains. Smart contracts\\ will be written in JavaScript.\end{tabular}                  \\ \hline
    \end{tabular}
    \caption{Technologies Used}
    \label{tab:technology}
\end{table}
Throughout Thesis A, I have undertaken a
deep-dive into Hyperledger Fabric in an attempt to better understand
its novel approach to smart-contract programming. I have analysed the
white papers for Hyperledger and understand the philosophy and
technological theory behind the framework. Parallel to doing
Thesis A I have done a deep-dive on blockchain technologies and
smart-contract programming on the Ethereum blockchain.
Although not a primary aim of this thesis, I have analysed
the economics of emission trading schemes and the difficulties
in encouraging long-term adoption of a carbon market.

\section{Preliminary Token Auctions}
To better understand auction applications on the blockchain, I
created a collection of Ethereum smart contracts for auctioning
underlying assets. Although my thesis
will not be making use of Ethereum or the Solidity programming
language, I thought the exercise would be useful
in providing an instruction on `event-driven' programming and
the performance of smart-contracts. The motivating example
was allowing students to bid for classes at university.

\begin{table}[H]
    \centering
    \begin{tabular}{||c c c ||}
        \hline
        No & Operation                   & Gas Cost \\ [0.5ex]
        \hline\hline
        1  & Add Admin                   & 43755    \\
        \hline
        2  & Add Lecturer                & 43893    \\
        \hline
        3  & Add Student                 & 43826    \\
        \hline
        4  & Add Transactor              & 43766    \\
        \hline
        5  & Transactor Token Transfer   & 23969    \\
        \hline
        6  & Set UOC Fee                 & 42536    \\
        \hline
        7  & Set Round Deadline          & 116664   \\
        \hline
        8  & Make Course                 & 2850856  \\
        \hline
        9  & Set Quota                   & 43642    \\
        \hline
        10 & Set Prerequisite for Course & 75069    \\
        \hline
        11 & Place Bid                   & 184511   \\
        \hline
        12 & Give Student Prerequisite   & 51786    \\ [1ex]
        \hline
    \end{tabular}
    \caption{Gas Cost for University}
    \label{tab:gascost}
\end{table}

An observation of Table~\ref{tab:gascost} shows the cost of
driving an auction on the blockchain comes from placing a bid
- an action costing up to 184,511 gas. The
example shows the difficulty in efficiently creating an auction
environment on the blockchain with an arbitrary number $n$ of bidders.
Using my research into pricing assets using blockchain tokens
with an auction, I decided to make the suggested model of an auction
for pricing
in Jiang et al a stretch goal for the project and not a primary aim.
The focus of the thesis is the creation of an emissions trading application
using Hydrogen certificates as a motivating example, and not the
efficient pricing of assets on a blockchain. Time permitting, I
will implement economically efficient pricing for carbon tokens
using an auctioning mechanism
similar to the application I created for Ethereum.