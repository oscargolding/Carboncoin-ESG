\chapter{Introduction}\label{ch:intro}
A blockchain can be defined as an immutable ledger maintained by a network of 
mutually untrusting peers. Since the creation of Bitcoin (BTC) in January 2009 
by the maverick Satoshi Nakamoto there has been an explosion of interest in 
the underlying technology behind blockchains. Specifically, the 
immutability and openness of the distributed ledger makes blockchain an 
attractive option for markets with untrusting participants exposed to 
information asymmetry. The market for carbon is particularly well suited to 
the blockchain due to inconsistent government policy - for example the 
ill-fated outcome of the Carbon Pollution Reduction Scheme in Australia. 

Recent attempts to create blockchain-based carbon markets have been met 
with some success - but have been held back by technicalities or innovations 
that disrupt the fundamental goal of using the blockchain as a ‘trust machine’. 
I will outline how hydrogen certificates on the blockchain can be used to 
automatically spend carbon tokens and add extra validation before being 
sold on the commodity market. Hydrogen is particularly well-suited as an 
example for certificate-based carbon markets due to hydrogen producers 
attaching a carbon footprint to certificates. 

I will outline how permissioned blockchains are principally useful for 
carbon markets due to support for high-throughput transactions. 
A carbon market would have to scale to a large number of distributed 
producers with scalability matching hydrogen energy production - 
an industry expected to be worth USD155 billion by 2022. I will propose 
Hyperledger Fabric as a blockchain framework - chiefly due to its support 
for up to 3500 transactions per second (TPS) and a plugable consensus 
algorithm. Moreover, my proposed solution will accommodate an optional 
carbon ‘reputation’ as part of the price for exchanging carbon tokens 
between producers. 

In Chapter~\ref{ch:intro} an overview of the thesis and its novel 
approach to certificate-based carbon markets is presented. 
In Chapter~\ref{ch:review} I outline the background to the project and 
important literature for markets on a blockchain. In Chapter~\ref{ch:plan} 
I provide methodology for the blockchain architecture. 
In Chapter~\ref{ch:prep} I outline some preliminary results with 
creating smart contracts (programs on the blockchain) for auctions on a 
public blockchain. In Chapter~\ref{ch:conclusion} I conclude with some 
remarks on the future potential for the thesis topic. 


