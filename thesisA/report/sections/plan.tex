\chapter{Project Plan}\label{ch:plan}
The decomposition of the project can be
divided into three parts:
a user interface for interacting with the
carbon market, a \textit{RESTful} HTTP server for
interfacing with Fabric and smart contracts on the
blockchain. An observation of Figure~\ref{fig:arch} details
how the certificate-based carbon market operates from a
high-level overview.

\begin{figure}[ht]
    \centering
    \begin{adjustbox}{max totalsize={2.1\textwidth}{1\textheight},center}
        \includesvg[inkscapelatex=false]{photos/arch.svg}
    \end{adjustbox}
    \caption{Certificate-based Carbon Market Architecture}
    \label{fig:arch}
\end{figure}

\section{Smart Contracts}
The smart contract component in the Fabric blockchain will
make extensive use of the following architectural
patterns: token template, token registry, policy contract
and burned token~\cite{patterns}. The token template pattern
will allow for the creation of a \textit{Carbon Token} which
is transferred between hydrogen producers who are looking
to purchase and sell tokens. The burned token pattern will
allow for tokens to be burned upon supplying a hydrogen
certificate. After the token is burned, the on-chain
smart contract for the emissions trading scheme marks
the particular certificate as `spent'. Token
registry will track which individuals have ownership of
particular tokens and will allow tokens to be transferred
between different owners. The token policy pattern
specifies how the associated level of hydrogen attached to
a hydrogen certificate is spent on carbon tokens within
the on-chain emissions trading contract.

\subsection{Token Transfers}
Hydrogen producers will be allowed to transfer tokens
between each other at an agreed price. The market
for transferring tokens will operate through a producer
creating an offer for the sale of a quantity of tokens
at a given price per token. A hydrogen producer looking
to purchase more tokens will purchase a number of tokens from
the seller at a given quantity. As observed in
Algorithm~\ref{alg:offer}, the purchaser of tokens
has the ability to consume a certain number of tokens from
an offer at an agreed upon price per token.

\begin{algorithm}
    \caption{Offer Acceptance}
    \begin{algorithmic}[1]
        \Procedure{AcceptOffer}{\textit{sellerId}, \textit{buyerId},
            \textit{offerId}, \textit{price}}
        \State let tS be \textit{tokens(sellerId)}
        \State let price be \textit{getOfferPrice(offerId)}
        \If{quantity $>$ tS}
        reject
        \EndIf
        \If{\textit{!hasFunds(buyerId, quantity, price)}}
        reject
        \EndIf
        \State \textit{swapTokens(buyerId, sellerId, quantity)}
        \EndProcedure
    \end{algorithmic}
    \label{alg:offer}
\end{algorithm}

\subsection{Token Burning}
The user can submit a hydrogen certificate to the
on-chain emission trading scheme smart contract. The smart-contract will check if the hydrogen certificate
has previously been supplied to the system. If the
certificate has not been submitted, then the smart-contract
maps the emissions in the certificate to the carbon tokens
contained inside the hydrogen producers carbon account. If the
user has a sufficient number of carbon tokens, then the tokens
will be subtracted from the account balance and `burned'.

\subsection{Direct Token Purchases}
The user will be allowed to purchase tokens from the
emissions trading scheme smart contract. Recently, literature
has been proposed allowing for the
construction of a game-theoretic approach to
pricing within blockchain energy markets~\cite{JIANG2020115239}.
The economic intution behind the pricing model is the
construction of a \textit{Stackelberg Competition} environment
where consumers continually outbid another until
equilibrium is achieved. Since the aim of the thesis
is the technical implementation of an emissions trading
blockchain, such a pricing feature would be out of scope.
Initially, users will be allowed to purchase tokens from the
emissions trading smart contract at a set price. As a
stretch goal, an auction environment will be constructed
allowing for a \textit{Stackelberg} competition
environment.

\section{HTTP Server}
The HTTP server will provide API endpoints to interface
with the Fabric blockchain. The Fabric SDK will be
extensively used on the API layer. The server will
employ the \textit{express} JavaScript framework
for \textit{Node.js}. The use-case for an HTTP Server
comes from the requirement for a user-interface to
interact with the Fabric blockchain.

\section{User Interface}
The user interface will allow for hydrogen energy producers
to interact with the Fabric blockchain to
purchase, sell and spend carbon tokens. The
user interface will be created using Facebook's
\textit{React} framework for use in browsers on a
mobile or desktop device. A collection of epic
user stories are supplied in Table~\ref{tab:estories}.
Epic user stories number one to seven are considered
fundamental to the aims and outcomes of the project.
Epic user story number eight allows for the creation of
an auction environment for the purchasing of carbon
tokens at a low price from the carbon token authority,
and will be a stretch goal for the thesis.

\begin{table}[ht]
    \centering
    \resizebox{\textwidth}{!}{
        \begin{tabular}{|l|l|l|}
            \hline
            \textbf{No} & \textbf{Epic User Story}   & \textbf{Description}       \\ \hline
            1           & \begin{tabular}[c]{@{}l@{}}As a hydrogen producer, I would like to directly \\ purchase more carbon tokens to emit more carbon.\end{tabular}  & \begin{tabular}[c]{@{}l@{}}A producer wants to purchase tokens \\ directly from the emissions trading \\ scheme.\end{tabular}  \\ \hline
            2           & \begin{tabular}[c]{@{}l@{}}As a hydrogen producer, I would like to sell \\ carbon tokens on a secondary market to remove\\ excess carbon token resources.\end{tabular}  & \begin{tabular}[c]{@{}l@{}}A producer wants to sell excess tokens \\ on a secondary market.\end{tabular}  \\ \hline
            3           & \begin{tabular}[c]{@{}l@{}}As a hydrogen producer, I would like to \\ purchase carbon tokens from a token sale offer\\ to emit carbon more cost-efficiently.\end{tabular}  & \begin{tabular}[c]{@{}l@{}}A producer wants to purchase carbon \\ tokens - preferably at a price \\ cheaper than what is offered \\ on the primary market.\end{tabular}  \\ \hline
            4           & \begin{tabular}[c]{@{}l@{}}As a hydrogen producer, I would like to spend \\ carbon tokens on energy certificates to \\ meet my environmental commitments.\end{tabular}  & \begin{tabular}[c]{@{}l@{}}Spend carbon tokens on energy \\ certificates - allows a producer \\ to subsequently trade the \\ hydrogen asset in a commodity market.\end{tabular} \\ \hline
            5           & \begin{tabular}[c]{@{}l@{}}As a hydrogen producer, I would like to \\ view all hydrogen sale offers to find the \\ most cost-efficient offer.\end{tabular} & \begin{tabular}[c]{@{}l@{}}Producer wants to find the most \\ cost-efficient carbon sale offer.\end{tabular} \\ \hline
            6           & \begin{tabular}[c]{@{}l@{}}As a hydrogen producer, I would like to observe\\ the number of carbon tokens inside my account \\ so I can plan for future Hydrogen production.\end{tabular} & \begin{tabular}[c]{@{}l@{}}Hydrogen producer would like to \\ view the number of tokens to plan for \\ future production.\end{tabular} \\ \hline
            7           & \begin{tabular}[c]{@{}l@{}}As a hydrogen producer, I would like to filter \\ carbon sale offers based on carbon reputation \\ to find the most environmentally friendly \\ offers.\end{tabular} & \begin{tabular}[c]{@{}l@{}}Environmentally conscious\\ producer wants to filter offers \\ according to the carbon \\ reputation of sellers.\end{tabular} \\ \hline
            8           & \begin{tabular}[c]{@{}l@{}}As a hydrogen producer, I would like to bid \\ in a carbon token auction to find low \\ prices for carbon emissions.\end{tabular} & \begin{tabular}[c]{@{}l@{}}An auction will be offered for \\ producers to purchase tokens at \\ timed intervals - mechanism of \\ adding carbon token inflation.\end{tabular} \\ \hline
        \end{tabular}
    }
    \caption{Epic User Stories for Producer Interaction}
    \label{tab:estories}
\end{table}

\section{Deployment}
Deployment of the application is suitable for a cloud
provider offering technical solutions for blockchain applications.
Amazon Web Services (AWS) offers a product called
\textit{Blockchain on AWS} allowing for the deployment
of permissioned blockchains (including Hyperledger Fabric)
on high-performance compute environments. Moreoever, the AWS
service \textit{Amplify} would allow for the hosting of
the user interface for Hydrogen producers to interact with the
Fabric blockchain. The HTTP server would be suitable for
running on an AWS service such as \textit{Lambda} or
\textit{EC2} depending on the scale of growth in the
hydrogen production market. The load-balancing features of
\textit{EC2} would help to deal with a rapidly growing
hydrogen market, with an unknown number of participants.

\section{Thesis Timeline}
The thesis timeline outlined in Table~\ref{tab:timeline}
shows the sequential nature of project. The primary
step, which also consumes the most effort will be smart
contract programming using Hyperledger Fabric. The smart
contracts will be programmed using JavaScript, thereby
unifying the entire technical stack under a single programming
lanaguage. I have undertaken research into Hyperledger
Fabric and installed the \textit{fabric samples} on my
development machine. The effort estimation for smart
contract programming is high due to the work involved in
understanding the Fabric SDK and the relevant libraries
for getting \textit{chaincode} to execute on Fabric.
My approach for completing the implementation of the thesis
is to focus on the low-level technicalities first: the completion of the \textit{express} API happens
after the \textit{chaincode} for the carbon market is
constructed. The smart contract programming will take
up to six weeks as a result of the learning curve
associated with Fabric. The API construction will take
up to three weeks to achieve, since I have some experience
with the \textit{express} framework. The construction of the
user interface for the carbon market will take a further three
weeks, as I also have experience with the creation of
applications using \textit{React}. Any time remaining will
be spent on making improvements to the system along with the
attempting of stretch goals related to blockchain pricing models.
\begin{table}[ht]
    \centering
    \resizebox{\textwidth}{!}{
        \begin{tabular}{|l|l|l|l|}
            \hline
            Week & Thesis A                                              & Thesis B                                           & Thesis C                                            \\ \hline
            1    &                                                       & \cellcolor[HTML]{C0C0C0}                           & \cellcolor[HTML]{C0C0C0}                            \\ \hline
            2    &                                                       & \cellcolor[HTML]{C0C0C0}                           & \cellcolor[HTML]{C0C0C0}                            \\ \hline
            3    & \cellcolor[HTML]{9B9B9B}Blockchain patterns           & \cellcolor[HTML]{C0C0C0}                           & \cellcolor[HTML]{C0C0C0}                            \\ \hline
            4    & \cellcolor[HTML]{C0C0C0}                              & \cellcolor[HTML]{9B9B9B}Smart contract programming & \cellcolor[HTML]{9B9B9B}User interface finalisation \\ \hline
            5    & \cellcolor[HTML]{C0C0C0}                              & \cellcolor[HTML]{C0C0C0}                           & \cellcolor[HTML]{C0C0C0}                            \\ \hline
            6    & \cellcolor[HTML]{C0C0C0}                              & \cellcolor[HTML]{C0C0C0}                           & \cellcolor[HTML]{C0C0C0}                            \\ \hline
            7    & \cellcolor[HTML]{9B9B9B}Ethereum Auction Example      & \cellcolor[HTML]{C0C0C0}                           & \cellcolor[HTML]{9B9B9B}\begin{tabular}[c]{@{}l@{}}Improvements and \\ stretch goal\end{tabular}  \\ \hline
            8    & \cellcolor[HTML]{C0C0C0}                              & \cellcolor[HTML]{9B9B9B}API construction           & \cellcolor[HTML]{C0C0C0}                            \\ \hline
            9    & \cellcolor[HTML]{9B9B9B}Problem domain and definition & \cellcolor[HTML]{C0C0C0}                           & \cellcolor[HTML]{C0C0C0}                            \\ \hline
            10   & \cellcolor[HTML]{9B9B9B}Literature Review             & \cellcolor[HTML]{C0C0C0}                           & \cellcolor[HTML]{C0C0C0}                            \\ \hline
            11   & \cellcolor[HTML]{C0C0C0}                              & \cellcolor[HTML]{C0C0C0}                           & \cellcolor[HTML]{C0C0C0}                            \\ \hline
            12   & \cellcolor[HTML]{C0C0C0}                              & \cellcolor[HTML]{C0C0C0}                           & \cellcolor[HTML]{C0C0C0}                            \\ \hline
            13   & \cellcolor[HTML]{C0C0C0}                              & \cellcolor[HTML]{C0C0C0}                           & \cellcolor[HTML]{C0C0C0}                            \\ \hline
        \end{tabular}
    }
    \caption{Timeline for Thesis}
    \label{tab:timeline}
\end{table}